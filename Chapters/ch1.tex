\section[Introduction]{\hyperlink{toc}{Introduction}}
\subsection*{Cosmology FAQ}
\addcontentsline{toc}{subsection}{\protect\numberline{}Cosmology FAQ}
There are no problems in the Introduction chapter in Ryden. However, this seems like the optimal place to address a bunch of commonly-asked questions in Cosmology.
\begin{enumerate}[(a)]
  \item Where is the center of the Universe?

  \textit{Answer:} Everywhere; the universe is homogenous on large scales, so there is no preferred location as a center!

  \item Why is the Universe expanding?

  \textit{Answer:} Because it was expanding yesterday (and it was expanding yesterday due to inflation).

  \item What is the Universe expanding into?

  \textit{Answer:} Itself; there is no boundary to the Universe. 

  % Remark 1: A spherically curved Universe is finite in volume, but has no edge.
  % Remark 2: A closed/finite Universe has to be completely electrically neutral. Electric field lines go out from postive charges, either ending ``at infinity'' or at a negative charge. In a closed universe, they can't go out to infinity; so has to end at a negative charge. Thus pairing the positive charge with a negative one!

  \item Why isn't the Solar System expanding?

  \textit{Answer:} It actually was in the past, but it has stopped now.

  \item Is the Universe rotating?

  \textit{Answer:} There are modified FRW metrics that allow for special axes (note this breaks the cosmological principle; if the Universe is described by these metrics, it would be homogenous but not isotropic). However, a rotating universe would generate spiral patterns on the CMB; the fact that we do not observe such patterns allows us to constrain global rotation.

  \item What happened before the Big Bang?

  \textit{Answer:} We don't know! There are theoretical models (such as cyclic Universes), but to a degree this is an unanswerable question.

  % Fun story: Isaac Asimov would ask people the question ``What happens when an unstoppable force meets an immovable object''? A bunch of people brushed this off as stupid. At a party, he asked a woman this question, who said it was unanswerable; apparently Asimov appreciated this answer so much that they got married!

  % Minutephysics would say that these two objects are really the same thing; that they are both impossible to accelerate. If we are to be consistent with their definitions, they must pass through each other.

  \item How come the CMB photons haven't outrun the galaxies since the Big Bang?

  \textit{Answer:} The Big Bang doesn't work that way; the Universe did not start at one point and then expand. It happened everywhere at once! CMB photons are emitted isotropically, from everywhere; so at every point in time we receive CMB photons that were further away.

  \item If the Universe is only 14 billion years old, how come we can see objects that are 40 billion light years away?

  \textit{Answer:} Due to the expansion of the universe, the horizon of the observable universe is not $ct_0$ but rather $\sim 3ct_0$. You can do the integrals to get the exact prefactor.

  \item What is dark matter?

  \textit{Answer:} A substance that makes up 29\% of our universe and only interacts through the gravitational force. We have a better picture of what it isn't than what it is.

  \item What will happen in the far future of the Universe?

  \textit{Answer:} We don't know! But in our best current model (the Benchmark model), a heat death is predicted.
\end{enumerate}
