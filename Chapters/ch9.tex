\section[Nucleosynthesis and the Early Universe]{\hyperlink{toc}{Nucleosynthesis and the Early Universe}}

\subsection{}
\subsection{}
\subsection{}
\subsection{}
\subsection{Neutrino Detection}
The cross-section for the interaction of a neutrino with a proton or neutron is:
\begin{equation}
    \sigma_w \sim 10^{-47}\si{m^2}\left(\frac{kT}{1\si{MeV}}\right)^2.
\end{equation}
So for a typical CNB neutrino with $E_\nu \sim kT_\nu \sim 5 \times 10^{-4}\si{eV}$, we have:
\begin{equation}
    \boxed{\sigma_w \sim 2.5 \times 10^{-66}\si{m^2}}
\end{equation}
Fe-56 has 26 protons, 26 electrons, and 30 neutrons per atom. It has a per-atom weight of:
\begin{equation}
    M = 26m_p + 26m_e + 30m_n \approx 56m_p = 56 \times 1.67 \times 10^{-27}\si{kg} =  9.34 \times 10^{-26}\si{kg}
\end{equation}
So the atomic number density is:
\begin{equation}
    n_a = \frac{\rho}{M} = \frac{7900\si{kg.m^{-3}}}{9.34 \times 10^{-26}\si{kg}} = 8.5 \times 10^{28}\si{m^{-3}}
\end{equation}
The number density of protons/electrons is therefore:
\begin{equation}
    \boxed{n_p = n_e = 26n_a = 2.2 \times 10^{30}\si{m^{-3}}}
\end{equation}
and the number density of neutrons is therefore:
\begin{equation}
    \boxed{n_n = 30n_a = 2.5 \times 10^{30}\si{m^{-3}}}
\end{equation}
The total number density of sub-atomic particles is therefore:
\begin{equation}
    n \sim n_p + n_e + n_n = 6.9 \times 10^{30}\si{m^{-3}}
\end{equation}
though of course this is approximate; really all the subatomic particles are clustered in their respective atoms. The mean free path of a CNB neutrino in this medium is given by:
\begin{equation}
    \boxed{\lambda = \frac{1}{n\sigma_w} = 5.8 \times 10^{34}\si{m}}
\end{equation}
\emph{Extremely} large; neutrinos are very hard to detect!