\section[Nucleosynthesis and the Early Universe]{\hyperlink{toc}{Nucleosynthesis and the Early Universe}}

\subsection{}
\subsection{}
\subsection{}
\subsection{}
\subsection{Neutrino Detection}
The cross-section for the interaction of a neutrino with a proton or neutron is:
\begin{equation}
    \sigma_w \sim 10^{-47}\si{m^2}\left(\frac{kT}{1\si{MeV}}\right)^2.
\end{equation}
So for a typical CNB neutrino with $E_\nu \sim kT_\nu \sim 5 \times 10^{-4}\si{eV}$, we have:
\begin{equation}
    \boxed{\sigma_w \sim 2.5 \times 10^{-66}\si{m^2}}
\end{equation}
Fe-56 has 26 protons, 26 electrons, and 30 neutrons per atom. It has a per-atom weight of:
\begin{equation}
    M = 26m_p + 26m_e + 30m_n \approx 56m_p = 56 \times 1.67 \times 10^{-27}\si{kg} =  9.34 \times 10^{-26}\si{kg}
\end{equation}
So the atomic number density is:
\begin{equation}
    n_a = \frac{\rho}{M} = \frac{7900\si{kg.m^{-3}}}{9.34 \times 10^{-26}\si{kg}} = 8.5 \times 10^{28}\si{m^{-3}}
\end{equation}
The number density of protons/electrons is therefore:
\begin{equation}
    \boxed{n_p = n_e = 26n_a = 2.2 \times 10^{30}\si{m^{-3}}}
\end{equation}
and the number density of neutrons is therefore:
\begin{equation}
    \boxed{n_n = 30n_a = 2.5 \times 10^{30}\si{m^{-3}}}
\end{equation}
The total number density of sub-atomic particles is therefore:
\begin{equation}
    n \sim n_p + n_e + n_n = 6.9 \times 10^{30}\si{m^{-3}}
\end{equation}
though of course this is approximate; really all the subatomic particles are clustered in their respective atoms. The mean free path of a CNB neutrino in this medium is given by:
\begin{equation}
    \boxed{\lambda = \frac{1}{n\sigma_w} = 5.8 \times 10^{34}\si{m}}
\end{equation}
\emph{Extremely} large; neutrinos are very hard to detect!

\subsection*{Optical Depth of Reionized Material (9.5 in first ed.)}
\addcontentsline{toc}{subsection}{\protect\numberline{}Optical Depth of Reionized Material (9.5 in first ed.)}
\begin{tcolorbox}
    We know from ovbservations that the intergalatic medium is currently ionized. Thus, at some point between $t_{\text{rec}}$ and $t_0$, the integalactic medium must have been reionized. In fact detailed measurements of the CMB on large scales place constraints on the amount of reionization (but that isn't important for this question). Assume that the baryonic component of the Universe instantaneously became completely reionized at some time $t_*$. For what value of $t_*$ does the optical depth of the reionized material:
    \begin{equation}
        \tau = \int_{t_*}^{t_0}\Gamma(t)\mathrm d t = \int_{t_*}^{t_0} n_e(t)\sigma_e c \mathrm d t
    \end{equation}
    equal one? For simplicity, assume that the Universe is spatially flat and matter-dominated, and that the baryonic component of the universe is pure hydrogen. To what redshift $z_*$ does this alue of $t_*$ correspond?
\end{tcolorbox}
\noindent
As the baryon density scales as $\propto \frac{1}{a^3}$, we have that:
\begin{equation}
    n_{e}(t)\sigma_e c = \frac{n_{e, 0}\sigma_e c}{a^3}
\end{equation}
Assuming that the baryonic component of the universe is pure hydrogen and that the universe is charge neutral, we have:
\begin{equation}
    n_{e, 0} = n_{\text{bary}, 0}
\end{equation}
For a flat, matter-dominated universe, we have scale factor (Ryden 5.5):
\begin{equation}
    a(t) = \left(\frac{t}{t_0}\right)^{2/3}
\end{equation}
so we find:
\begin{equation}
    n_{e}(t)\sigma_e c = \frac{n_{\text{bary}, 0}\sigma_e c t_0^2}{t^2}
\end{equation}
so carrying out the integral we have:
\begin{equation}
    1 = \tau = \int_{t_0}^{t_*}\frac{n_{\text{bary}, 0}\sigma_e c t_0^2}{t^2} \mathrm d t = n_{\text{bary}, 0}\sigma_e c t_0^2\left(\frac{1}{t_0} - \frac{1}{t_*}\right)
\end{equation}
So rearranging for $t_*$ we have:
\begin{equation}
    t_* = \frac{1}{\frac{1}{t_0} - \frac{1}{n_{\text{bary}, 0}\sigma_ec t_0^2}}
\end{equation}
In a flat, matter-dominated universe, the age of the universe is given by:
\begin{equation}
    t_0 = \frac{2}{3H_0}
\end{equation}
so we obtain:
\begin{equation}
    t_* = \frac{1}{\frac{3H_0}{2} - \frac{9H_0^2}{4n_{\text{bary}, 0}\sigma_ec}}
\end{equation}
Therefore, taking $H_0 = 68 \si{km.s^{-1}.Mpc^{-1}}$, $n_{\text{bary}, 0} = 0.25\si{m^{-3}}$, $c = 3 \times 10^{8}\si{m.s^{-1}}$, and $\sigma_e = 6.65 \times 10^{-29}\si{m^2}$, we find:
\begin{equation}
    \boxed{t_* = 4.2 \times 10^{14}\si{s} = 13 \si{Myr}}
\end{equation}