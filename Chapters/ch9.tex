\section[Nucleosynthesis and the Early Universe]{\hyperlink{toc}{Nucleosynthesis and the Early Universe}}

\subsection{Mass fraction of Helium with faster decay}
After the proton/neutron freezeout, the ratio of neutrons to protons is approximately:
\begin{equation}
    f_0 = \frac{n_{n, 0}}{n_{p, 0}} \approx \frac{1}{5}
\end{equation}
Suppose the time delay until nucleosynthesis is $t$. In this time delay, the neutrons decay to $\exp(-t/\tau_n)$ of their original amount, and the protons increase by the amount the neutrons decay (as the neutrons decay into a proton and electron). Therefore, the neutron-to-proton to ratio as a function of delay time is:
\begin{equation}\label{ntop}
    f(t) = \frac{n_n(t)}{n_p(t)} = \frac{n_{n, 0}\exp(-t/\tau_n)}{n_{p, 0} + n_{n, 0}(1 - \exp(-t/\tau_n))} = \frac{\frac{n_{n, 0}}{n_{p, 0}}\exp(-t/\tau_n)}{1 + \frac{n_{n, 0}}{n_{p, 0}}(1 - \exp(-t/\tau_n))} = \frac{f_0\exp(-t/\tau_n)}{1 + f_0(1 - \exp(-t/\tau_n))}
\end{equation}
The time delay from freezeout until nucleosynthesis is $200\si{s}$, and we suppose that the neutron decay time is reduced to $\tau_n = 88\si{s}$, so we can compute the fraction $f(200)$ at the time of nucleosynthesis to be:
\begin{equation}
    f(200) = 0.017
\end{equation}
If we assume that all available neutrons are incorporated into Helium, we get the maximal value for the primordial Helium fraction (as derived in problem 9.4) so:
\begin{equation}
    \boxed{Y_{\text{max}} = \frac{2f(200)}{1+f(200)} = 0.033}
\end{equation}

\subsection{Mass fraction of Helium with different rest energies}
Note that technically, this new difference in the mass fraction would likely cause some difference in the binding energy of the deuteron. This would affect the temperature $T_{\text{nuc}}$ of nucleosynthesis, as:
\begin{equation}
    1 \approx 6.5\eta\left(\frac{kT_{\text{nuc}}}{m_n c^2}\right)^{3/2}\exp(\frac{B_{D}}{kT_{\text{nuc}}})
\end{equation}
and therefore the nucleosynthsis time $t_{\text{nuc}}$ would also change. For simplicity's sake, let us assume that $B_D$ remains unchanged. However, the difference in the mass energy \emph{will} affect the neutron-to-proton ratio at freezeout, where we find:
\begin{equation}
    f_0 = \frac{n_{n, 0}}{n_{p, 0}} = \exp(-\frac{Q_{n}}{kT_{\text{freeze}}})
\end{equation}
so with $Q_{n} = 0.129\si{MeV}$ instead of $1.29\si{MeV}$ and $kT_{\text{freeze}} = 0.8\si{MeV}$, we find:
\begin{equation}
    f_0 = 0.85.
\end{equation}
So, using \eqref{ntop} with $f_0 = 0.85$, $t = 200\si{s}$, $\tau_n = 880\si{s}$, we find:
\begin{equation}
    f(200) = 0.58
\end{equation}
therefore the maximum possible mass fraction is given by:
\begin{equation}
    \boxed{Y_{\text{max}} = \frac{2f(200)}{1 + f(200)} = 0.734}
\end{equation}


\subsection{Helium Increase in Our Galaxy (ADD LATER)}

\subsection{Maximum value for primordial Helium Fraction}
By definition, we have:
\begin{equation}\label{Ymaxdef}
    Y_{\text{max}} = \frac{\rho_{\text{He}}}{\rho_{\text{b}}}
\end{equation}
Further, assuming that all neutrons are contained in Helium (as we want the maximal primordial Helium fraction), we have:
\begin{equation}
    \rho_{\text{He}} = 2\rho_{n}
\end{equation}
And the Baryonic density is given by:
\begin{equation}
    \rho_b = \rho_n + \rho_p
\end{equation}
Where:
\begin{equation}
    \rho_n = m_nn_n, \quad \rho_p = m_pn_p
\end{equation}
So substituting these into \eqref{Ymaxdef} we have:
\begin{equation}
    Y_{\text{max}} = \frac{2\rho_n}{\rho_n + \rho_p} = \frac{2\rho_n/\rho_p}{1 + \rho_n/\rho_p} = \frac{2\frac{n_n}{n_p}\frac{m_n}{m_p}}{1 + \frac{n_n}{n_p}\frac{m_n}{m_p}}
\end{equation}
Defining $f = \frac{n_n}{n_p}$ and making the approximation that $\frac{m_n}{m_p} \approx 1$, we conclude:
\begin{equation}
    \boxed{Y_{\text{max}} \approx \frac{2f}{1 + f}}
\end{equation}

\subsection{Neutrino Detection}
The cross-section for the interaction of a neutrino with a proton or neutron is:
\begin{equation}
    \sigma_w \sim 10^{-47}\si{m^2}\left(\frac{kT}{1\si{MeV}}\right)^2.
\end{equation}
So for a typical CNB neutrino with $E_\nu \sim kT_\nu \sim 5 \times 10^{-4}\si{eV}$, we have:
\begin{equation}
    \boxed{\sigma_w \sim 2.5 \times 10^{-66}\si{m^2}}
\end{equation}
Fe-56 has 26 protons, 26 electrons, and 30 neutrons per atom. It has a per-atom weight of:
\begin{equation}
    M = 26m_p + 26m_e + 30m_n \approx 56m_p = 56 \times 1.67 \times 10^{-27}\si{kg} =  9.34 \times 10^{-26}\si{kg}
\end{equation}
So the atomic number density is:
\begin{equation}
    n_a = \frac{\rho}{M} = \frac{7900\si{kg.m^{-3}}}{9.34 \times 10^{-26}\si{kg}} = 8.5 \times 10^{28}\si{m^{-3}}
\end{equation}
The number density of protons/electrons is therefore:
\begin{equation}
    \boxed{n_p = n_e = 26n_a = 2.2 \times 10^{30}\si{m^{-3}}}
\end{equation}
and the number density of neutrons is therefore:
\begin{equation}
    \boxed{n_n = 30n_a = 2.5 \times 10^{30}\si{m^{-3}}}
\end{equation}
The total number density of sub-atomic particles is therefore:
\begin{equation}
    n \sim n_p + n_e + n_n = 6.9 \times 10^{30}\si{m^{-3}}
\end{equation}
though of course this is approximate; really all the subatomic particles are clustered in their respective atoms. The mean free path of a CNB neutrino in this medium is given by:
\begin{equation}
    \boxed{\lambda = \frac{1}{n\sigma_w} = 5.8 \times 10^{34}\si{m}}
\end{equation}
\emph{Extremely} large; neutrinos are very hard to detect!

\subsection*{Optical Depth of Reionized Material (9.5 in first ed.)}
\addcontentsline{toc}{subsection}{\protect\numberline{}Optical Depth of Reionized Material (9.5 in first ed.)}
\begin{tcolorbox}
    We know from ovbservations that the intergalatic medium is currently ionized. Thus, at some point between $t_{\text{rec}}$ and $t_0$, the integalactic medium must have been reionized. In fact detailed measurements of the CMB on large scales place constraints on the amount of reionization (but that isn't important for this question). Assume that the baryonic component of the Universe instantaneously became completely reionized at some time $t_*$. For what value of $t_*$ does the optical depth of the reionized material:
    \begin{equation}
        \tau = \int_{t_*}^{t_0}\Gamma(t)\mathrm d t = \int_{t_*}^{t_0} n_e(t)\sigma_e c \mathrm d t
    \end{equation}
    equal one? For simplicity, assume that the Universe is spatially flat and matter-dominated, and that the baryonic component of the universe is pure hydrogen. To what redshift $z_*$ does this alue of $t_*$ correspond?
\end{tcolorbox}
\noindent
As the baryon density scales as $\propto \frac{1}{a^3}$, we have that:
\begin{equation}
    n_{e}(t)\sigma_e c = \frac{n_{e, 0}\sigma_e c}{a^3}
\end{equation}
Assuming that the baryonic component of the universe is pure hydrogen and that the universe is charge neutral, we have:
\begin{equation}
    n_{e, 0} = n_{\text{bary}, 0}
\end{equation}
For a flat, matter-dominated universe, we have scale factor (Ryden 5.5):
\begin{equation}
    a(t) = \left(\frac{t}{t_0}\right)^{2/3}
\end{equation}
so we find:
\begin{equation}
    n_{e}(t)\sigma_e c = \frac{n_{\text{bary}, 0}\sigma_e c t_0^2}{t^2}
\end{equation}
so carrying out the integral we have:
\begin{equation}
    1 = \tau = \int_{t_0}^{t_*}\frac{n_{\text{bary}, 0}\sigma_e c t_0^2}{t^2} \mathrm d t = n_{\text{bary}, 0}\sigma_e c t_0^2\left(\frac{1}{t_0} - \frac{1}{t_*}\right)
\end{equation}
So rearranging for $t_*$ we have:
\begin{equation}
    t_* = \frac{1}{\frac{1}{t_0} - \frac{1}{n_{\text{bary}, 0}\sigma_ec t_0^2}}
\end{equation}
In a flat, matter-dominated universe, the age of the universe is given by:
\begin{equation}
    t_0 = \frac{2}{3H_0}
\end{equation}
so we obtain:
\begin{equation}
    t_* = \frac{1}{\frac{3H_0}{2} - \frac{9H_0^2}{4n_{\text{bary}, 0}\sigma_ec}}
\end{equation}
Therefore, taking $H_0 = 68 \si{km.s^{-1}.Mpc^{-1}}$, $n_{\text{bary}, 0} = 0.25\si{m^{-3}}$, $c = 3 \times 10^{8}\si{m.s^{-1}}$, and $\sigma_e = 6.65 \times 10^{-29}\si{m^2}$, we find:
\begin{equation}
    \boxed{t_* = 4.2 \times 10^{14}\si{s} = 13 \si{Myr}}
\end{equation}