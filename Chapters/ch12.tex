\section[Structure Formation: Baryons and Photons]{\hyperlink{toc}{Structure Formation: Baryons and Photons}}

\subsection{Galaxies more luminous than $L$}
We start with the Schechter luminosity function which gives the number density $\Phi(L)\mathrm d L$ in the range $L \rightarrow L + \mathrm d L$:
\begin{equation}
    \Phi(L) \mathrm d L = \Phi^*\left(\frac{L}{L^*}\right)^\alpha\exp(-\frac{L}{L^*})\frac{\mathrm d L}{L^*}
\end{equation}
The number density of galaxies more luminous than $L$ is given by integrating from $L$ to $\infty$:
\begin{equation}
    n_{\geq L} = \int_L^\infty \Phi(L')\mathrm d L' = \int_L^\infty \Phi^*\left(\frac{L'}{L^*}\right)^\alpha\exp(-\frac{L'}{L^*})\frac{\mathrm d L'}{L^*}
\end{equation}
Now substituting $t = \frac{L'}{L^*}$ which gives $\mathrm d t = \frac{\mathrm d L'}{L^*}$ we find:
\begin{equation}
    n_{\geq L} = \Phi^* \int_{L/L^*}^\infty t^\alpha e^{-t} \mathrm d t
\end{equation}
Furthermore, we can recognize the RHS as an incomplete gamma function, yielding:
\begin{equation}
    \boxed{n_{\geq L} = \Phi^* \Gamma(\alpha + 1, \frac{L}{L^*})}
\end{equation}
In the limit $L \to 0$, we just have the regular gamma function:
\begin{equation}
    n_{\geq 0} = n_{\text{tot}} = \Phi^* \int_0^\infty t^\alpha e^{-t}\mathrm d t = \Phi^*\Gamma(\alpha + 1)
\end{equation}
But with $\alpha = -1$, the integral:
\begin{equation}
    n_{\text{tot}} =\int_0^\infty \frac{e^{-t}}{t}\mathrm d t = \infty
\end{equation}
is seen to diverge (at the lower limit). The physical solution is as follows; in this limit, the number density of galaxies diverges, but the luminosity density remains finite (so the average luminosity per galaxy is zero); this can be realized via the following calculation:
\begin{equation}
    \Psi_{\text{tot}} = \int_0^\infty L'\Phi(L')\mathrm d L = \int_0^\infty L'\Phi^*\left(\frac{L'}{L^*}\right)^{-1}\exp(-\frac{L'}{L^*})\frac{\mathrm d L'}{L^*} = 
    \Phi^*\int_0^\infty \exp(-\frac{L'}{L^*})\mathrm d L' = \Phi^*L^*.
\end{equation}

\subsection{Total Luminosity Density}
We again tart with the Schechter luminosity function:
\begin{equation}
    \Phi(L) \mathrm d L = \Phi^*\left(\frac{L}{L^*}\right)^\alpha\exp(-\frac{L}{L^*})\frac{\mathrm d L}{L^*}
\end{equation}
We can now integrate $L \Phi(L)\mathrm d L$ from $0$ to $\infty$ to obtain the total luminosity density:
\begin{equation}
    \Psi = \int_0^\infty \Phi(L') \mathrm d L' = \int_0^\infty L'\Phi^*\left(\frac{L'}{L^*}\right)^{\alpha}\exp(-\frac{L'}{L^*})\frac{\mathrm d L'}{L^*} = \int_0^\infty \Phi^*\left(\frac{L'}{L^*}\right)^{\alpha+1}\exp(-\frac{L'}{L^*})\mathrm d L'
\end{equation}
Again making the substitution $t = \frac{L'}{L^*}$ we find:
\begin{equation}
    \Psi = \int_0^\infty \Phi^*L^* t^{\alpha+1} e^{-t}\mathrm d t = \Phi^*L^*\int_0^\infty t^{\alpha + 1}e^{-t}\mathrm d t
\end{equation}
We recognize the rightmost expression as a gamma function, which yields:
\begin{equation}
    \boxed{\Psi = \Phi^*L^*\Gamma(\alpha + 2)}
\end{equation}
Now, observing that $\Gamma(-1 + 2) = \Gamma(1) = 1$, so we find for the $V$ band that:
\begin{equation}
    \boxed{\Psi_V = \Phi^*L^*_V\Gamma(1) = 1 \times 10^8 \si{L_{\odot, V} Mpc^{-3}}}
\end{equation}

\subsection{Do structures of $10^{17}M_\odot$ exist today?}
From Ryden Eq. 12.38, the total amount of mass inside the last scattering surface is given by:
\begin{equation}
    M_{\text{tot}} \approx 4.3 \times 10^{23}M_\odot
\end{equation}
Dividing this by $M = 10^{17}M_\odot$, we find:
\begin{equation}
    N = \frac{M_{\text{tot}}}{M} = 4.3 \times 10^6 \si{regions}
\end{equation}
The very first $M = 10^{17}M_\odot$ structure to collapse is the one region out of 4.3 million that had the highest overdensity at the time of radiation-matter equality, with probability:
\begin{equation}
    P = \frac{1}{N} = 2.3 \times 10^{-7}.
\end{equation}
This is equivalent to a $5.04\sigma$ deviation in a Gaussian distribution. Since $\sigma = \delta M/M = 0.12$, then we can compute the redshift of collapse to be:
\begin{equation}
    1 + z_{\text{coll}} = 5.04\sigma = 0.6048
\end{equation}
i.e. the first such object has not begin to collapse (as $z_{\text{coll}} < 0$) and hence $\boxed{\text{we do not}}$ expect to see gravitationally collapsed structures with mass $M = 10^{17}M_\odot$ today.

\subsection{Ripping Apart Galaxies}
An object is ripped apart when its energy density $\e_m$ becomes less than the phantom energy density $\e_p$. Hence we can set these two densities to be equal:
\begin{equation}
    \e_m = \e_p.
\end{equation}
We know that $\e_p \propto a^{-3(1+w_p)}$, and in particular:
\begin{equation}
   \e_p = \Omega_{p, 0}\e_c a^{-3(1+w_p)}.
\end{equation}
So solving for the rip-apart scale factor $a$ we find:
\begin{equation}
    a = \left(\frac{\e_m}{\Omega_p \e_c}\right)^{-1/(3+3w_p)}
\end{equation}
and for $w_p = -1.1$ this becomes:
\begin{equation}
    a = \left(\frac{\e_m}{\Omega_p \e_c}\right)^{10/3}.
\end{equation}
For the milky way galaxy, we have $M_{\text{gal}} = 9.6 \times 10^{11}M_\odot$ and $R_{\text{gal}} = 75\si{kpc}$. Assuming a uniform mass density and using that $\rho_c = 8.7 \times 10^{-27}\si{kg.m^{-3}}$, we find:
\begin{equation}
    \frac{\e_\text{gal}}{\e_c} = \frac{\rho_{\text{gal}}}{\rho_c} = \frac{M_{\text{gal}}}{\frac{4}{3}\pi R_{\text{gal}}^3 \rho_c} = 4.2 \times 10^{3}
\end{equation}
Therefore the scale factor for which the Milky way galaxy would be ripped apart can be solved to be:
\begin{equation}
    \boxed{a_{\text{gal}} = 4.0 \times 10^{12}}
\end{equation} 
We can do the same for the sun, with mass $M_\odot$ and radius $R_\odot = 7 \times 10^8 \si{m}$ to find:
\begin{equation}
    \frac{\e_{\text{sun}}}{\e_c} = \frac{\rho_{\text{sun}}}{\rho_c} = \frac{M_\odot}{\frac{4}{3}\pi R_\odot^3 \rho_c} = 1.6 \times 10^{29}
\end{equation}
so the scale factor for which the sun would be ripped apart would be:
\begin{equation}
    \boxed{a_{\text{sun}} = 7.2 \times 10^{97}}
\end{equation}

From Problem 5.5, we know the time between the present time $t_0$ and the Big rip $t_{\text{rip}}$ was found for $H_0 = 68\si{km.s^{-1}.Mpc^{-1}}$, $w_p = -1.1$ and $\Omega_{m, 0} = 0.3$ to be:
\begin{equation}
    t_{\text{rip}} - t_0 = 115.5\si{Gyr}
\end{equation}
with $t_0 = 13.8 \si{Gyr}$, we can find $t_{\text{rip}}$ to be:
\begin{equation}
    t_{\text{rip}} = 115.5\si{Gyr} - 13.8\si{Gyr} = 101.7\si{Gyr}
\end{equation}
Further, we can use the intermediate result from problem 5.5 (Eq. \eqref{int55}) that:
\begin{equation}
    \int_{t_0}^{t_{\text{rip}}}H_0 dt  \approx \int_1^\infty a^{3(1+w_p)/2}(1 - \Omega_{m, 0})^{-1/2}\mathrm d a = \frac{1}{\sqrt{\Omega_{p, 0}}}\int_1^\infty a^{3(1+w_p)/2-1}\mathrm d a
\end{equation}
Now, we can replace $t_0$ with $t_{\text{gal}}$ (the rip-apart time for the Milky Way Galaxy), and $a(t_0) = 1$ with $a(t_{\text{gal}}) = a_\text{gal}$ in the above equation, and carry out the integral to get:
\begin{equation}
    H_0(t_{\text{rip}} - t_{\text{gal}}) = \frac{1}{\sqrt{\Omega_{p, 0}}}\left.\frac{2}{3(1+w_p)}a^{3(1+w_p)/2}\right|_{a_{\text{gal}}}^\infty
\end{equation}
The term at infinity vanishes (as the exponent of $a$ is negative), and we therefore find:
\begin{equation}
    t_{\text{gal}} = t_{\text{rip}} - \frac{1}{H_0\sqrt{\Omega_{p, 0}}}\frac{2}{3\abs{1 + w_p}}a_{\text{gal}}^{3(1+w_p)/2}
\end{equation}
Numerically, this evaluates to:
\begin{equation}
    \boxed{t_{\text{gal}} = 101.7\si{Gyr} - 0.46\si{Gyr} = 101.w\si{Gyr}}
\end{equation}
So we find that the Milky way will be ripped apart about half a gigayear before the big rip. We can do the same with $t_{\text{sun}}$ and $a(t_\text{sun}) = a_{\text{sun}}$ to find:
\begin{equation}
    \boxed{t_{\text{sun}} = 101.7\si{Gyr} - 7.5\times 10^{-14}\si{Gyr} \approx 101.7\si{Gyr}}
\end{equation}
In other words, the sun will not be ripped apart until the big rip.



