\section[Structure Formation: Gravitational Instability]{\hyperlink{toc}{Structure Formation: Gravitational Instability}}

\subsection{Density Fluctuations in a Flat, Matter-Dominated, Contracting Universe}
We first note that:
\begin{equation}
    H = \frac{\dot{a}}{a} < 0
\end{equation}
in a contracting universe. Now, starting with Ryden Eq. 11.49, we have:
\begin{equation}
    \ddot{\delta} + 2H\dot{\delta} - \frac{3}{2}\Omega_m H^2\delta = 0
\end{equation}
In a contracting matter dominated universe, $\Omega_m = 1$ and $H = -\frac{2}{3t}$ (note the negative sign for contraction! This can be obtained from $a(t) = \left(\frac{t}{t_0}\right)^{2/3}$), so:
\begin{equation}
    \ddot{\delta} - \frac{4}{3t}\dot{\delta} - \frac{2}{3t^2}\delta = 0
\end{equation}
Guessing a power law $\delta(t) = t^r$, we find:
\begin{equation}
    r(r-1)t^{r-2} - \frac{4}{3t}rt^{r-1} - \frac{2}{3t^2}t^r = 0
\end{equation}
Dividing both sides by $t^{r-2}$ we obtain a quadratic equation for $r$:
\begin{equation}
    r^2 - \frac{7}{3}r - \frac{2}{3} = 0
\end{equation}
Using the quadratic formula, we find the solutions:
\begin{equation}
    r = \frac{7 \pm \sqrt{73}}{6}
\end{equation}
So we therefore find:
\begin{equation}
    \boxed{\delta(t) = At^{\frac{7 + \sqrt{73}}{6}} + Bt^{\frac{7 - \sqrt{73}}{6}}}
\end{equation}
Where $A, B$ are constants determined by initial conditions. The second term vanishes for large $t$, so:
\begin{equation}
    \boxed{\delta(t) \approx At^{\frac{7 + \sqrt{73}}{6}}}
\end{equation}

\subsection{Density Fluctuations in a Nearly Empty, Negatively Curved, Expanding Universe}
We again start with Ryden Eq. 11.49:
\begin{equation}
    \ddot{\delta} + 2H\dot{\delta} - \frac{3}{2}\Omega_m H^2\delta = 0
\end{equation}
Since $\Omega_m \ll 1$, the last term can be neglected:
\begin{equation}
    \ddot{\delta} + 2H\dot{\delta} = 0
\end{equation}
In an empty expanding universe, we have $H = \frac{1}{t}$ (as $a(t) = \frac{t}{t_0}$) so:
\begin{equation}
    \ddot{\delta} + \frac{2}{t}\dot{\delta} = 0
\end{equation}
Again guessing a power law $\delta(t) = t^r$, we find:
\begin{equation}
    r(r-1)t^{r-2} + \frac{2}{t}rt^{r-1} = 0
\end{equation}
Dividing both sides by $t^{r-2}$ we obtain a quadratic equation for $r$:
\begin{equation}
    r^2 + r = 0
\end{equation}
This has solutions:
\begin{equation}
    r = 0, r = -1
\end{equation}
So we therefore find:
\begin{equation}
    \boxed{\delta(t) = At^0 + Bt^{-1}}
\end{equation}
At long times, the latter terms vanishes, so:
\begin{equation}
    \boxed{\delta(t) \approx A}.
\end{equation}
That is, at long times the fluctuations are constant.

\subsection{Photon-Baryon Fluid}
For the photon-baryon fluid, we have:
\begin{equation}
    \dod{P}{\e} = \dod{P}{a}\dod{a}{\e} = \dod{P}{a}\left(\dod{\e}{a}\right)^{-1}
\end{equation}
Since $P = P_\gamma = \frac{1}{3}\e_{\gamma} = \frac{1}{3}\e_{\gamma, 0} a^{-4}$, we find:
\begin{equation}
    \dod{P}{a} = -\frac{4}{3}\e_{\gamma, 0} a^{-5}
\end{equation}
Furthermore, since $\e = \e_\gamma + \e_{\text{bary}} = \e_{\gamma, 0}a^{-4} + \e_{\text{bary}, 0}a^{-3}$ we find:
\begin{equation}
    \dod{\e}{a} = -4\e_{\gamma, 0}a^{-5} - 3\e_{\text{bary}, 0}a^{-4}
\end{equation}
Therefore:
\begin{equation}
    \boxed{\dod{P}{\e} = \frac{-\frac{4}{3}\e_{\gamma, 0} a^{-5}}{-4\e_{\gamma, 0}a^{-5} - 3\e_{\text{bary}, 0}a^{-4}} = \frac{1}{3}\left(\frac{1}{1 + \frac{3}{4}\frac{\e_{\text{bary}, 0}}{\e_{\gamma,0}} a}\right)}
\end{equation}. Now, we can find the sound speed to be:
\begin{equation}\label{cs}
    \boxed{c_s = c\left(\dod{P}{\e}\right)^{1/2} = c\sqrt{\frac{1}{3}\left(\frac{1}{1 + \frac{3}{4}\frac{\e_{\text{bary}, 0}}{\e_{\gamma,0}} a}\right)}}
\end{equation}
In Ryden Eq. 11.26, the Jeans length of the photon-baryon fluid neglecting the contribution from the baryons was estimated to be:
\begin{equation}
    \lambda_{J, \gamma} \approx 3\frac{c}{H(z_{\text{dec}})} \approx 0.66\si{Mpc}
\end{equation}
We can replace $c$ with $c_s$ in the above expression to determine $\lambda_J$ accounting for the baryons. Before this, we briefly resolve some remaining parameters in our expression for $c_s$. $z_{\text{dec}} = 1090$, so:
\begin{equation}
    a_{\text{dec}} = \frac{1}{1 + z_{\text{dec}}} = 9.17 \times 10^{-4}
\end{equation}
Furthermore, from the benchmark model we find the energy density ratio of:
\begin{equation}
    \frac{\e_{\text{bary}, 0}}{\e_{\gamma,0}} = \frac{\Omega_{\text{bary, 0}}}{\Omega_{\gamma, 0}} = \frac{0.048}{5.35\times 10^{-5}} = 897
\end{equation}
Plugging these values of $a_{\text{dec}}, \frac{\e_{\text{bary}, 0}}{\e_{\gamma,0}}$ into \eqref{cs}, we find:
\begin{equation}
    c_s = 0.45c
\end{equation}
compared to the $c_s = \frac{c}{\sqrt{3}} = 0.577c$ prediction neglecting baryons. Hence the Jeans length being proportional to $c_s$ gets scaled down by this ratio:
\begin{equation}
    \lambda_{J, \text{bary} + \gamma} = \frac{0.45c}{0.577c} \lambda_{J, \gamma} = 0.51\si{Mpc}
\end{equation}
So we were off by:
\begin{equation}
    \boxed{\lambda_{J, \gamma} - \lambda_{J, \text{bary} + \gamma} = 0.15\si{Mpc}}
\end{equation}

\subsection{Milky Way Gravitational Collapse}
For any circular orbit (as we can assume for a spherical dark halo), the orbital speed is related to $R_{\text{halo}}$ and $M_{\text{gal}}$ by (Ryden Eq. 7.9):
\begin{equation}\label{vcircular}
    v^2 = \frac{GM_{\text{gal}}}{R_{\text{halo}}}
\end{equation}
Further assuming a uniform distribution of mass throughout the galaxy, we have:
\begin{equation}\label{rhouniform}
    \rho = \frac{M_{\text{gal}}}{V} = \frac{M_{\text{gal}}}{\frac{4}{3}\pi R_{\text{halo}}^3}
\end{equation}
Combining \eqref{vcircular} and \eqref{rhouniform}, we obtain:
\begin{equation}
    \rho = \frac{\frac{R_{\text{halo}} v^2}{G}}{\frac{4}{3}\pi R_{\text{halo}}^3} = \frac{v^2}{\frac{4}{3}\pi G R_{\text{halo}}^2}
\end{equation}
Substituting this into $t_{\text{min}} \approx \frac{1}{\sqrt{G\rho}}$ we find:
\begin{equation}
    t_{\text{min}} \approx \frac{1}{\sqrt{G\frac{v^2}{\frac{4}{3}\pi G R_{\text{halo}}^2}}} = \sqrt{\frac{3}{4\pi}}\frac{R_{\text{halo}}}{v}
\end{equation}
So we therefore conclude:
\begin{equation}
    \boxed{t_{\text{min}} \sim \frac{R_{\text{halo}}}{v}}
\end{equation}
For our own Galaxy, we take $R_{\text{halo}} \approx 75\si{kpc}$ and $v = 235\si{km.s^{-1}}$ to find:
\begin{equation}
    \boxed{t_{\text{min}} \approx 9.86 \times 10^{15}\si{s} = 312\si{Myr}}
\end{equation}
$t_{\text{min}}$ defines the maximum possible redshift in which we could see galaxies with comparable to $v$ and $R_{\text{min}}$; in other words we find $z(t_{\text{min}})$. Neglecting the early radiation period, we can take our universe to be dominated by matter and the cosmological constant. For this Universe, the Friedmann equation has the analytic solution (Ryden Eq. 5.101):
\begin{equation}
    H_0 t = \frac{2}{3\sqrt{1 - \Omega_{m, 0}}}\ln\left[\left(\frac{a}{a_{m\Lambda}}\right)^{3/2} + \sqrt{1 + \left(\frac{a}{a_{m\Lambda}}\right)^3}\right]
\end{equation}
With the Benchmark models of $H_0 = 68 \si{km.s^{-1}.Mpc^{-1}}$, $\Omega_{m, 0} = 0.31$, and $a_{m\Lambda} = 0.77$, and the pervious result that $t = t_{\text{min}} = 312\si{Myr}$, we can numerically solve for $a$ in the above relation. Doing so, we find:
\begin{equation}
    a_{\text{min}} = 0.069
\end{equation}
From which we obtain the maximum redshift:
\begin{equation}
    \boxed{z_{\text{max}} = \frac{1}{a_{\text{min}}} - 1 = 13.5}
\end{equation}

\subsection{Coma Cluster Gravitational Collapse}
We first note (from Ryden 7.37) that:
\begin{equation}
    M_{\text{Coma}} = \frac{\avg{v^2}r_h}{\alpha G}
\end{equation}
Taking again $\rho$ to be uniform, we find:
\begin{equation}
    \rho = \frac{M_{\text{Coma}}}{V} = \frac{\frac{\avg{v^2}r_h}{\alpha G}}{\frac{4}{3}\pi r_h^3} = \frac{3\avg{v^2}}{4\pi\alpha G r_h^2}
\end{equation}
So using that $t_{\text{min}} \approx \frac{1}{\sqrt{G\rho}}$ we find:
\begin{equation}
    t_{\text{min}} \approx \frac{1}{\sqrt{G \frac{3\avg{v^2}}{4\pi\alpha G r_h^2}}} = \sqrt{\frac{4\pi \alpha}{3}}\frac{r_h}{\sqrt{\avg{v^2}}} \sim \frac{r_h}{\sqrt{\avg{v^2}}} 
\end{equation}
So numerically we find: 
\begin{equation}
    \boxed{t_{\text{min}} \approx 3.05 \times 10^{16}\si{s} = 967\si{Myr}}
\end{equation}

\subsection{Mean Square Mass Fluctuation and Standard Deviation of Density Field}
The standard deviation of the density field for a Gaussian field can be computed (in the case of a Gaussian field) as (Ryden Eq. 11.67):
\begin{equation}
    \sigma_\delta^2 = \frac{V}{2\pi}\int_0^\infty P(k)k^2 \mathrm dk = \frac{V}{2\pi}\int_0^{k_{max}} P(k)k^2 \mathrm dk
\end{equation}
Meanwhile the mean square mass fluctuation is computed as:
\begin{equation}
    \avg{\left(\frac{\delta M}{M}\right)^2} = \avg{\left(\frac{M  - \avg{M}}{\avg{M}}\right)^2} = \frac{V}{2\pi }\int_0^{\infty}P(k)\left[\frac{3j_1(kr)}{kr}\right]^2 k^2 \mathrm d k = \frac{V}{2\pi }\int_0^{k_{\text{max}}}P(k)\left[\frac{3j_1(kr)}{kr}\right]^2 k^2 \mathrm d k 
\end{equation}
So the claim is proven is we can show for $M < M_{\text{min}}$, or for $r < r_{\text{min}}$ that:
\begin{equation}
    \int_{0}^{k_{max}} P(k)k^2 \mathrm d k = \int_0^{k_{\text{max}}}P(k)\left[\frac{3j_1(kr)}{kr}\right]^2 k^2 \mathrm d k 
\end{equation}
Incoming is a handwavey argument that I don't actually think is the full answer. For $r < r_{\text{min}} = \frac{2\pi}{k_{\text{max}}}$, $kr$ will be small over the domain of integration so we may Taylor expand $j_1(kr)$. Doing so, we find:
\begin{equation}
    j_1(kr) = \frac{\sin(kr) - kr\cos(kr)}{(kr)^2} \approx \frac{(kr - \frac{(kr)^3}{6}) - kr\left(1 - \frac{(kr)^2}{2}\right)}{(kr)^2} = \frac{\frac{(kr)^3}{3}}{(kr)^2} = \frac{kr}{3}
\end{equation}
Therefore:
\begin{equation}
    \int_0^{k_{\text{max}}}P(k)\left[\frac{3j_1(kr)}{kr}\right]^2 k^2 \mathrm d k \approx  \int_0^{k_{\text{max}}}P(k)\left[\frac{3}{kr}\frac{kr}{3}\right]^2 k^2 \mathrm d k = \int_0^{k_{\text{max}}}P(k) k^2 \mathrm d k
\end{equation}
Which is exactly what we wished to show.