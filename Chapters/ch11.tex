\section[Structure Formation: Gravitational Instability]{\hyperlink{toc}{Structure Formation: Gravitational Instability}}

\subsection{Density Fluctuations in a Flat, Matter-Dominated, Contracting Universe}
We first note that:
\begin{equation}
    H = \frac{\dot{a}}{a} < 0
\end{equation}
in a contracting universe. Now, starting with Ryden Eq. 11.49, we have:
\begin{equation}
    \ddot{\delta} + 2H\dot{\delta} - \frac{3}{2}\Omega_m H^2\delta = 0
\end{equation}
In a contracting matter dominated universe, $\Omega_m = 1$ and $H = -\frac{2}{3t}$ (note the negative sign for contraction! This can be obtained from $a(t) = \left(\frac{t}{t_0}\right)^{2/3}$), so:
\begin{equation}
    \ddot{\delta} - \frac{4}{3t}\dot{\delta} - \frac{2}{3t^2}\delta = 0
\end{equation}
Guessing a power law $\delta(t) = t^r$, we find:
\begin{equation}
    r(r-1)t^{r-2} - \frac{4}{3t}rt^{r-1} - \frac{2}{3t^2}t^r = 0
\end{equation}
Dividing both sides by $t^{r-2}$ we obtain a quadratic equation for $r$:
\begin{equation}
    r^2 - \frac{7}{3}r - \frac{2}{3} = 0
\end{equation}
Using the quadratic formula, we find the solutions:
\begin{equation}
    r = \frac{7 \pm \sqrt{73}}{6}
\end{equation}
So we therefore find:
\begin{equation}
    \boxed{\delta(t) = At^{\frac{7 + \sqrt{73}}{6}} + Bt^{\frac{7 - \sqrt{73}}{6}}}
\end{equation}
Where $A, B$ are constants determined by initial conditions. The second term vanishes for large $t$, so:
\begin{equation}
    \boxed{\delta(t) \approx At^{\frac{7 + \sqrt{73}}{6}}}
\end{equation}

\subsection{Density Fluctuations in a Nearly Empty, Negatively Curved, Expanding Universe}
We again start with Ryden Eq. 11.49:
\begin{equation}
    \ddot{\delta} + 2H\dot{\delta} - \frac{3}{2}\Omega_m H^2\delta = 0
\end{equation}
Since $\Omega_m \ll 1$, the last term can be neglected:
\begin{equation}
    \ddot{\delta} + 2H\dot{\delta} = 0
\end{equation}
In an empty expanding universe, we have $H = \frac{1}{t}$ (as $a(t) = \frac{t}{t_0}$) so:
\begin{equation}
    \ddot{\delta} + \frac{2}{t}\dot{\delta} = 0
\end{equation}
Again guessing a power law $\delta(t) = t^r$, we find:
\begin{equation}
    r(r-1)t^{r-2} + \frac{2}{t}rt^{r-1} = 0
\end{equation}
Dividing both sides by $t^{r-2}$ we obtain a quadratic equation for $r$:
\begin{equation}
    r^2 + r = 0
\end{equation}
This has solutions:
\begin{equation}
    r = 0, r = -1
\end{equation}
So we therefore find:
\begin{equation}
    \boxed{\delta(t) = At^0 + Bt^{-1}}
\end{equation}
At long times, the latter terms vanishes, so:
\begin{equation}
    \boxed{\delta(t) \approx A}.
\end{equation}
That is, at long times the fluctuations are constant.

\subsection{Photon-Baryon Fluid (ADD LATER)}
\subsection{Milky Way Gravitational Collapse (ADD LATER)}

\subsection{Coma Cluster Gravitational Collapse}
We first note (from Ryden 7.37) that:
\begin{equation}
    M_{\text{Coma}} = \frac{\avg{v^2}r_h}{\alpha G}
\end{equation}
Taking again $\rho$ to be uniform, we find:
\begin{equation}
    \rho = \frac{M_{\text{Coma}}}{V} = \frac{\frac{\avg{v^2}r_h}{\alpha G}}{\frac{4}{3}\pi r_h^3} = \frac{3\avg{v^2}}{4\pi\alpha G r_h^2}
\end{equation}
So using that $t_{\text{min}} \approx \frac{1}{\sqrt{G\rho}}$ we find:
\begin{equation}
    t_{\text{min}} \approx \frac{1}{\sqrt{G \frac{3\avg{v^2}}{4\pi\alpha G r_h^2}}} = \sqrt{\frac{4\pi \alpha}{3}}\frac{r_h}{\sqrt{\avg{v^2}}} \sim \frac{r_h}{\sqrt{\avg{v^2}}} 
\end{equation}
So numerically we find: 
\begin{equation}
    \boxed{t_{\text{min}} \approx 3.05 \times 10^{16}\si{s} = 967\si{Myr}}
\end{equation}

\subsection{Mean Square Mass Fluctuation and Standard Deviation of Density Field}
The standard deviation of the density field for a Gaussian field can be computed (in the case of a Gaussian field) as (Ryden Eq. 11.67):
\begin{equation}
    \sigma_\delta^2 = \frac{V}{2\pi}\int_0^\infty P(k)k^2 \mathrm dk = \frac{V}{2\pi}\int_0^{k_{max}} P(k)k^2 \mathrm dk
\end{equation}
Meanwhile the mean square mass fluctuation is computed as:
\begin{equation}
    \avg{\left(\frac{\delta M}{M}\right)^2} = \avg{\left(\frac{M  - \avg{M}}{\avg{M}}\right)^2} = \frac{V}{2\pi }\int_0^{\infty}P(k)\left[\frac{3j_1(kr)}{kr}\right]^2 k^2 \mathrm d k = \frac{V}{2\pi }\int_0^{k_{\text{max}}}P(k)\left[\frac{3j_1(kr)}{kr}\right]^2 k^2 \mathrm d k 
\end{equation}
So the claim is proven is we can show for $M < M_{\text{min}}$, or for $r < r_{\text{min}}$ that:
\begin{equation}
    \int_{0}^{k_{max}} P(k)k^2 \mathrm d k = \int_0^{k_{\text{max}}}P(k)\left[\frac{3j_1(kr)}{kr}\right]^2 k^2 \mathrm d k 
\end{equation}
Incoming is a handwavey argument that I don't actually think is the full answer. For $r < r_{\text{min}} = \frac{2\pi}{k_{\text{max}}}$, $kr$ will be small over the domain of integration so we may Taylor expand $j_1(kr)$. Doing so, we find:
\begin{equation}
    j_1(kr) = \frac{\sin(kr) - kr\cos(kr)}{(kr)^2} \approx \frac{(kr - \frac{(kr)^3}{6}) - kr\left(1 - \frac{(kr)^2}{2}\right)}{(kr)^2} = \frac{\frac{(kr)^3}{3}}{(kr)^2} = \frac{kr}{3}
\end{equation}
Therefore:
\begin{equation}
    \int_0^{k_{\text{max}}}P(k)\left[\frac{3j_1(kr)}{kr}\right]^2 k^2 \mathrm d k \approx  \int_0^{k_{\text{max}}}P(k)\left[\frac{3}{kr}\frac{kr}{3}\right]^2 k^2 \mathrm d k = \int_0^{k_{\text{max}}}P(k) k^2 \mathrm d k
\end{equation}
Which is exactly what we wished to show.