\section[Inflation and the Very Early Universe]{\hyperlink{toc}{Inflation and the Very Early Universe}}

\subsection{Upper limit on Primordial Density (ADD LATER)}

\subsection{Solving the Monopole Problem}
If monopoles formed at the GUT time with one monopole per horizon of mass $m_M = m_{GUT}$, then the energy density of these monopoles would be given by:
\begin{equation}
    e_M(t_{GUT}) \approx \frac{m_Mc^2}{(2ct_{GUT})} = 10^{106}\si{eV.m^{-3}}
\end{equation}
where we take $m_Mc^2 \sim E_{GUT} \sim 10^{12}\si{TeV}$ and $t_{GUT} = 10^{-36}\si{s}$. Therefore the density parameter of the monopoles at this time is given by:
\begin{equation}
    \Omega_M(t_{GUT}) = \frac{\e_{M}(t_{GUT})}{\e_{c, 0}} \approx 10^{96}
\end{equation}
where we take $\e_{c, 0} = 5 \times 10^9\si{eV.m^{-3}}$. For a radiation-only universe, the scale factor goes as (Ryden Eq. 5.60):
\begin{equation}
    a(t) = \left(\frac{t}{t_0}\right)^{1/2}
\end{equation}
So at the GUT time:
\begin{equation}
    a(t_{GUT}) = \left(\frac{t_{GUT}}{t_0}\right)^{1/2}
\end{equation}
Taking $t_0 \sim 13.8\si{Gyr} = 4.35 \times 10^{17}\si{s}$ and $t_{GUT}$ as before, we find:
\begin{equation}
    a(t_{GUT}) = 1.5 \times 10^{-27}.
\end{equation}
Since the magnetic monopole density parameter should scale as $\frac{1}{a^3}$ with time (like regular matter), if inflation did not happen, the density of monopoles today would be:
\begin{equation}
    \Omega_M(t_0) = \Omega_M(t_{GUT})a(t_{GUT})^3 = 3.4 \times 10^{15}
\end{equation}
This is off from the observational limits by:
\begin{equation}
    \frac{\Omega_{M, 0, \text{observed}}}{\Omega_M(t_0)} = \frac{10^{-6}}{3.4 \times 10^{15}} = 2.9 \times 10^{-22}.
\end{equation}
To account for this, we must have had $N$ e-folds of inflation, leading the scale factor is actually $e^{N}$ larger than what we calculated above. Hence:
\begin{equation}
    2.9 \times 10^{-22} = \frac{1}{e^{3N}}
\end{equation}
which we can solve for $N$ to obtain:
\begin{equation}
    \boxed{N \approx 17}
\end{equation}


\subsection{False Vacuum}
The Universe dominated by the false vacuum has the same structure as one with dominated by a cosmological constant. In such a universe, the Hubble parameter s given by Ryden 5.72:
\begin{equation}
    \boxed{H_i = \left(\frac{8\pi G\e_{\Lambda}}{3c^2}\right)^{1/2} = 1.83 \times 10^{-18}\si{s^{-1}} = 0.83H_0}
\end{equation}
where we take $H_0$ is the Hubble constant in our universe. We assume that at the false vacuum decay time that the false vacuum decays into blackbody photons, so we can therefore use the blackbody radiation temperature to solve for what the temperature of the Universe would be at this time:
\begin{equation}
    \boxed{T = \sqrt[4]{\frac{\e_{\Lambda}}{\alpha}} = 29\si{K}}
\end{equation}
To find the energy density of matter at this time, we first compute the scale factor at this time; this is given by Ryden 5.73 to be:
\begin{equation}
    a(t_f) = a(t_0)e^{H_i(t_f - t_0)} = e^{H_i(t_f - t_0)} 
\end{equation}
where we take $a(t_0) = 1$ by convention. So with $H_i$ as above, $t_f = 50t_0$ and $t_0 = 13.7\si{Gyr}$, we find:
\begin{equation}
    a(t_f) = 6.7 \times 10^{16}
\end{equation}
So the energy density of matter at this time would be:
\begin{equation}
    \boxed{\e_{m}(t_f) = a(t_f)^{-3} \e_{m, 0} = 5 \times 10^{-48}\si{MeV.m^{-3}}}
\end{equation}
To find when the universe is again dominated by matter, we wish to find the time when:
\begin{equation}\label{matterradequality}
    \e_{m} = \e_r.
\end{equation}
Further, we know that:
\begin{equation}\label{matterradscaling}
    \e_m(a) = \frac{\e_{m, 0}}{a^3}, \quad \e_{r}(a) = \frac{\e_{r, 0}}{a^4}
\end{equation}
First solving for what $\e_{r, 0}$ would be, we have:
\begin{equation}
    \e_{r, 0} = \e_{r}(t_f)a(t_f)^4 = 6.7 \times 10^{70}\si{MeV.m^{-3}}
\end{equation}
Combining \eqref{matterradequality} and \eqref{matterradscaling} we find:
\begin{equation}
    \frac{\e_{m, 0}}{a_{rm}^3} = \frac{\e_{r, 0}}{a_{rm}^4}
\end{equation}
Hence:
\begin{equation}
    a_{rm} = \frac{\e_{r, 0}}{\e_{m, 0}}
\end{equation}
Further in a radiation dominated universe we know we have $a(t) = \left(\frac{t}{t_0}\right)^{1/2}$ (Ryden Eq. 5.60), so:
\begin{equation}
    \boxed{t_{rm} = t_0 \left(\frac{\e_{r, 0}}{\e_{m, 0}}\right)^2}
\end{equation}
Numerically we obtain:
\begin{equation}
    \boxed{t_{rm} = 2.7 \times 10^{136}\si{Gyr}}
\end{equation}

\subsection*{Gamow's CMB Prediction (10.3 in first ed.) (ADD LATER)}
\addcontentsline{toc}{subsection}{\protect\numberline{}Gamow's CMB Prediction (10.3 in first ed.) (ADD LATER)}
\begin{tcolorbox}
    A fascinating bit of cosmological history is that of George Gamow's prediction of the Cosmic Microwave Background in 1948. (Unfortunately, his prediction was premature; by the time the CMB was actually discovered in the 1960's, his prediction had fallen into obscurity.) Let's see if you can reproduce Gamow's line of argument. Gamow knew that nucleosynthesis must have taken place at a temperature $T_{\text {nuc }} \simeq 10^{9} \mathrm{~K}$, and that the age of the Universe is currently $t_{0} \simeq 10 \mathrm{Gyr}$. Assume that the Universe is flat and contains only radiation. With these assumptions, what was the energy density $\epsilon$ at the time of nucleosynthesis? What was the Hubble parameter $\mathrm{H}$ at the time of nucleosynthesis? What was the time $t_{\text {nuc }}$ at which nucleosynthesis took place? What is the current temperature $T_{0}$ of the radiation filling the Universe today? If the Universe switched from being radiation-dominated to being matter-dominated at a redshift $z_{r m}>0$, will this increase or decrease $T_{0}$ for fixed values of $T_{\text {nuc }}$ and $t_{0}$ ? Explain your answer.
\end{tcolorbox}