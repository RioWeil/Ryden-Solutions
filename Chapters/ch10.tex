\section[Inflation and the Very Early Universe]{\hyperlink{toc}{Inflation and the Very Early Universe}}

\subsection{}
\subsection{}
\subsection{}
\subsection*{Gamow's CMB Prediction (10.3 in first ed.)}
\addcontentsline{toc}{subsection}{\protect\numberline{}Gamow's CMB Prediction (10.3 in first ed.)}
\begin{tcolorbox}
    A fascinating bit of cosmological history is that of George Gamow's prediction of the Cosmic Microwave Background in 1948. (Unfortunately, his prediction was premature; by the time the CMB was actually discovered in the 1960's, his prediction had fallen into obscurity.) Let's see if you can reproduce Gamow's line of argument. Gamow knew that nucleosynthesis must have taken place at a temperature $T_{\text {nuc }} \simeq 10^{9} \mathrm{~K}$, and that the age of the Universe is currently $t_{0} \simeq 10 \mathrm{Gyr}$. Assume that the Universe is flat and contains only radiation. With these assumptions, what was the energy density $\epsilon$ at the time of nucleosynthesis? What was the Hubble parameter $\mathrm{H}$ at the time of nucleosynthesis? What was the time $t_{\text {nuc }}$ at which nucleosynthesis took place? What is the current temperature $T_{0}$ of the radiation filling the Universe today? If the Universe switched from being radiation-dominated to being matter-dominated at a redshift $z_{r m}>0$, will this increase or decrease $T_{0}$ for fixed values of $T_{\text {nuc }}$ and $t_{0}$ ? Explain your answer.
\end{tcolorbox}